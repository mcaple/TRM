% Please add the following required packages to your document preamble:
% \usepackage[table,xcdraw]{xcolor}
% If you use beamer only pass "xcolor=table" option, i.e. \documentclass[xcolor=table]{beamer}
%%\begin{table}[]
%%\begin{tabular}{|l|l|l|}
%%\hline
%%\rowcolor[HTML]{C0C0C0} 
%%{\color[HTML]{000000} No.} & {\color[HTML]{000000} Guidelines}            %%               & {\color[HTML]{000000} Explanation}                                                                                                                                                                                                                                                                                                                                                                                                                                                                                                                                                                                                                                                                                                                                                                                                               %%               \\ \hline
%%1                          & Produce a viable artefact                    %%               & \begin{tabular}[c]{@{}l@{}}Design-science research must produce a workable, practical artefact in the form of a construct, model, method, or instantiation.\\ The\end{tabular}                                                                                                                                                                                                                                                                                                                                                                                                                                                                                                                                                                                                                                                                                  \\ \hline
%%2                          & Ensure that the artefact produced is relevant and important & The artefact produced must assist with the resolution of a problem that is deemed relevant and important to some stakeholder community.                                                                                                                                                                                                                                                                                                                                                                                                                                                                                                                                                                                                                                                                                                                         \\ \hline
%%3                          & Rigorously evaluate the artefact produced                   & The effectiveness and efficiency of the artefact must be evaluated using rigorous methods. For instance, it might be evaluated analytically using a mathematical model or empirically using a field study or experiment. The evaluation of an artefact should also include ‘an element of style’, which reflects ‘human perception and taste’.                                                                                                                                                                                                                                                                                                                                                                                                                                                                                                                  \\ \hline
%%4                          & Produce an artefact that makes a research contribution      & \begin{tabular}[c]{@{}l@{}}The artefact produced must make a significant contribution to knowledge via the artefact itself, or the methods used to construct the artefact, or the methods used to evaluate the artefact. For this outcome to occur, the contribution to knowledge must be novel. Moreover, it will be easier to demonstrate a contribution to knowledge if the artefact provides a solution to a previously unsolved problem, or it is uncertain whether a working artefact can even be constructed, or the artefact’s ability to perform ‘appropriately’ is unclear.\\ The\end{tabular}                                                                                                                                                                                                                                                        \\ \hline
%%5                          & Follow rigorous construction methods                        & The artefact must be constructed in a rigorous way. In particular, construction methods must be sufficiently well specified and formalised for other researchers to be able to replicate the way it is constructed. Appropriate levels of rigour should be chosen, however, because excessive rigour can result in the relevance of the artefact being undermined (its usefulness to stakeholders is decreased).                                                                                                                                                                                                                                                                                                                                                                                                                                                \\ \hline
%%6                          & Show the artefact is the outcome of a search process        & \begin{tabular}[c]{@{}l@{}}The artefact should reflect the outcome of a search process whereby available means (actions and resources) are used to reach a desired end under the constraint of ‘laws’ that apply (natural or social laws). The current state of a system (e.g., the artefact being designed) is compared against a goal state. Actions are then taken (sometimes based on heuristics) to reduce the differences between the current and the goal states. The search for actions to reduce differences is iterative until an optimal or a satisfactory solution (match between the current and goal states) is found. To achieve tractable design solutions, the search process often involves simplification and abstraction of the means, ends, and laws and decomposition of the overall problem into simpler sub- problems.\\ 7\end{tabular} \\ \hline
%%7                          & Clearly communicate the research process and outcome        & The research process and outcome must be communicated clearly to stakeholders (both researchers and practitioners). Sufficient detail must be provided to enable (a) the artefact to be constructed and used effectively, and (b) the resources needed to build and use the artefact to be determined.                                                                                                                                                                                                                                                                                                                                                                                                                                                                                                                                                          \\ \hline
%%\end{tabular}
%%\end{table}


\begin{table}[ht]
  \begin{center}
    \caption{Hevner et al. 7 Guidelines for Design-Science Research}
    \label{tab:HevnerGuidelines}
    \begin{tabular}{l|l} % <-- Alignments: 1st column left, 2nd middle and 3rd right, with vertical lines in between
      \textbf{No.} & \textbf{Guideline}\\
      \hline
      1 & Produce a viable artefact\\
      \hline
      2 & Ensure that the artefact produced is relevant and important\\
      \hline
      3 & Rigorously evaluate the artefact produced\\
      \hline
      4 & Produce an artefact that makes a research contribution\\
      \hline
      5 & Follow rigorous construction methods\\
      \hline
      6 & Show the artefact is the outcome of a search process\\
      \hline
      7 & Clearly communicate the research process and outcome\\
      \hline
%      \multirow{2}{*}{Anomaly Detection} & anomaly, anomalous, imbalance, rarity, exception, \\
%      & oddity, inconsistency, abnormality\\
    \end{tabular}
  \end{center}
\end{table}




%\begin{figure}[htbp]
%\caption{Hevner et al. Guidelines for design-science research} %\label{fig:HevnerGuidelines}
%\begin{table}[]
%\begin{tabular}{|l|l|}
%\hline
%{No.} & {Guidelines}           \\ \hline
%1                          & Produce a viable artefact                     %              \\ \hline
%2                          & Ensure that the artefact produced is relevant %and important \\ \hline
%3                          & Rigorously evaluate the artefact produced     %              \\ \hline
%4                          & Produce an artefact that makes a research %contribution      \\ \hline
%5                          & Follow rigorous construction methods          %              \\ \hline
%6                          & Show the artefact is the outcome of a search %process        \\ \hline
%7                          & Clearly communicate the research process and %outcome        \\ \hline
%\end{tabular}
%\end{table}
%\end{figure}



%\begin{figure}[htbp]
%\caption{Hevner et al. Guidelines for design-science research} %\label{fig:HevnerGuidelines}

%\begin{enumerate}[leftmargin=*]
%    \item Produce a viable artefact%. Design-science research must produce a workable, practical artefact in the form of a construct, model, method, or instantiation.

%    \item Ensure that the artefact produced is relevant and important%. The artefact produced must assist with the resolution of a problem that is deemed relevant and important to some stakeholder community.

%    \item Rigorously evaluate the artefact produced%. The effectiveness and efficiency of the artefact must be evaluated using rigorous methods. For instance, it might be evaluated analytically using a mathematical model or empirically using a field study or experiment. The evaluation of an artefact should also include ‘an element of style’, which reflects ‘human perception and taste’.

%    \item Produce an artefact that makes a research contribution%. The artefact produced must make a significant contribution to knowledge via the artefact itself, or the methods used to construct the artefact, or the methods used to evaluate the artefact. For this outcome to occur, the contribution to knowledge must be novel. Moreover, it will be easier to demonstrate a contribution to knowledge if the artefact provides a solution to a previously unsolved problem, or it is uncertain whether a working artefact can even be constructed, or the artefact’s ability to perform ‘appropriately’ is unclear.

%    \item Follow rigorous construction methods%. The artefact must be constructed in a rigorous way. In particular, construction methods must be sufficiently well specified and formalised for other researchers to be able to replicate the way it is constructed. Appropriate levels of rigour should be chosen, however, because excessive rigour can result in the relevance of the artefact being undermined (its usefulness to stakeholders is decreased).

%    \item Show the artefact is the outcome of a search process%. The artefact should reflect the outcome of a search process whereby available means (actions and resources) are used to reach a desired end under the constraint of ‘laws’ that apply (natural or social laws). The current state of a system (e.g., the artefact being designed) is compared against a goal state. Actions are then taken (sometimes based on heuristics) to reduce the differences between the current and the goal states. The search for actions to reduce differences is iterative until an optimal or a satisfactory solution (match between the current and goal states) is found. To achieve tractable design solutions, the search process often involves simplification and abstraction of the means, ends, and laws and decomposition of the overall problem into simpler sub- problems.

%    \item Clearly communicate the research process and outcome%. The research process and outcome must be communicated clearly to stakeholders (both researchers and practitioners). Sufficient detail must be provided to enable (a) the artefact to be constructed and used effectively, and (b) the resources needed to build and use the artefact to be determined.
%\end{enumerate}

%\end{figure}
