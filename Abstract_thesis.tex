\thispagestyle{empty}
\begin{abstract}
This study attempts to review modern deep learning techniques for class imbalance in the hope of remedying such issues existing within a dataset from a company that UTS is currently collaborating with. Although a great deal of research studies are available into the class imbalance issue using traditional methods this paper also considers the area of deep learning. 
\par
Today's researchers in the field of machine learning are often judged, amongst other things, by the accuracy of their classification. However, this accuracy can be greatly affected when one class within a given dataset vastly outnumbers another. The result of such an imbalance creates a situation where most classifications favour the majority class and so the minority class becomes difficult if not impossible to detect. Although this sounds like an unwanted situation if we judge our classification purely on its accuracy we will have achieved a favourable result as most classifications will indeed be correct. Of course the researcher's credibility is not the only concern with this imbalance as many real world classification problems involve heavily imbalanced datasets. Imagine for instance, attempting to predict whether a patient has a malignant melanoma or a benign lesion. In such a scenario not being able to detect the greatly outnumbered malignant classification would obviously be catastrophic for the patient. This paper attempts to firstly categorise current accepted methods to reduce the imbalance problem and then to highlight the fact that most comparison studies that have been written to date have been applied to the field of computer vision. 
% This paper will apply all comparisons to non image datasets.  
% Finally some of the approaches will be applied to a specific industrial dataset, which has shown to contain a strong imbalance, in the hope that the findings from previous research aid to a more robust classification. 
%The company owning the dataset is currently collaborating with UTS in the hope of validating the use of machine learning within their business processes.       

%For this reason although predictive accuracy could be applied in a reasonably balanced dataset it is more usual to see an ROC curve or something similar applied in an imbalanced situation (Ling \& Li, 1998~\cite{ling1998data}; Drummond \& Holte, 2000~\cite{drummond2000explicitly} Provost \& Fawcett, 2001~\cite{provost2001robust}; Bradley 1997~\cite{bradley1997use}) .  

\vspace{8pt}
\noindent Dissertation directed by Professor FH \\
School of Computer Science
 
\end{abstract}
