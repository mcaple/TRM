\noindent
In order to confirm that this research is both novel and of relevance the methodology that it follows will be one of a systematic literature review (SLR).
\par
This approach was formalised with a set of guidelines produced by Kitchenham and Charters in 2007 and is very useful in confirming that no relevant work is excluded. In order to conduct a review the guidelines suggest the following stages

\begin{itemize}
  \item Identification of research %(See Section 6.1).
  \item Selection of primary studies %(See Section 6.2).
  \item Study quality assessment %(See Section 6.3).
  \item Data extraction and monitoring %(See Section 6.4).
  \item Data synthesis %(See Section 6.5).
\end{itemize}

\noindent
The following section's take each of these stages and describe how they have been applied for this research. Through using this approach we firstly highlight key research in the field and then show how this research benefits our understanding of the field and leads to our choice of research topic.


\subsection{Identification of research}
Identification of research is essential in confirming amongst other things that the researcher has not been biased in what he/she wants to find. It is this rigour that delineates SLR from other approaches.
\par
The following databases have been identified as representing a broad enough source for all relevant literature searches so as to nullify any question surrounding researcher bias.

\begin{enumerate}
  \item Scopus (\href{https://www.scopus.com/home.uri}{https://www.scopus.com/home.uri})
  \item Science \& Technology\\ (\href{https://www.proquest.com/libraries/government/science-technology/}{https://www.proquest.com/libraries/government/science-technology/})
  \item ACM Digital Library (\href{https://dl.acm.org/}{https://dl.acm.org/})
  \item Google Scholar (\href{https://scholar.google.com/}{https://scholar.google.com/})
  \item IEEE Explore (\href{https://ieeexplore.ieee.org/Xplore/home.jsp}{https://ieeexplore.ieee.org/Xplore/home.jsp})
\end{enumerate}

These databases were selected as they cover a wide corpus of the literature associated with Information Technology and in particular that of machine learning and artificial intelligence.

\begin{table}[h!]
  \begin{center}
    \caption{Search items and corresponding keywords}
    \label{tab:table1}
    \begin{tabular}{l|l} % <-- Alignments: 1st column left, 2nd middle and 3rd right, with vertical lines in between
      \textbf{Search item} & \textbf{Keyword}\\
      \hline
      Questionnaire & census, application, survey, questionnaire, poll, canvass\\
      \hline
      Closed Data & closed\\
      \hline
      Health & medical, health\\
      \hline
      Classification & classification\\
      \hline
      \multirow{2}{*}{Anomaly Detection} & anomaly, anomalous, imbalance, rarity, exception, \\
      & oddity, inconsistency, abnormality\\
    \end{tabular}
  \end{center}
\end{table}




This leads our search string to the following form \textit{("data mining" OR "classification")
  AND "association rules" AND ("questionnaire" OR "poll" OR "census" OR "canvass" OR "survey")
  AND ("fuzzy" OR "non crisp") }
% ("questionnaire" OR "poll" OR "census"  OR "canvass" OR "survey" ) AND
%   ("data mining" or classification) AND
%   ("medical" OR "health") AND
%   ("classification") AND
%   ("anomaly" OR "anomalous" OR "imbalance" OR "rarity" OR "exception" OR "oddity" OR "inconsistency" OR "abnormality")}

The large number of references produced from such searches will be recorded and managed through Mendeley Desktop.

Also any relevant information about the searches themselves will be recorded along with the search. This may include information such as date of search, years covered or any specific conditions relating to the search.

\subsection{Inclusions and Exclusions}

\noindent
Inclusion and exclusion criteria will initially be set to the following

\noindent
\textbf{Inclusion:}
\begin{enumerate}
  \item Available online
  \item Article is peer reviewed
  \item Full text is available in English
  \item Article on or after 2005
  \item Can be an academic or commercial project
        % \item CHANGE!!!! Does the article has a segment that argues about four, defined search categories (Table II): Big Data Sources, Big Data Ecosystem, Big data ecosystem challenges, and security and privacy?
\end{enumerate}

\noindent
\textbf{Exclusion:}
\begin{enumerate}
  \item Non English papers
  \item Duplicate studies
  \item Magazines, newspapers, websites, podcasts, blogs
\end{enumerate}

\noindent
The excluded resources will however be maintained so that if during the process of inclusion/exclusion too few works result are presented then the criteria maybe adjusted in order to include other relevant work.

\subsection{Selection of primary studies}

The query options for the \textit{1st Filter} for each of the selected databases will be included in this section along with the numbers of relevant papers discovered shown in Table~\ref{tab:SLR_Count}.
After the initial selection process \textit{2nd Filter} shows the number of relevant papers after a title review to see if the papers are clearly not relevant. \textit{3rd Filter} is after a review of the abstract to decide whether the paper is still suitable for the study.

\begin{table}[h!]
    \begin{center}
        \caption{Search count for SLR results from chosen databases}
        \label{tab:SLR_Count}
        \begin{tabular}{l|l|l|l} % <-- Alignments: 1st column left, 2nd middle and 3rd right, with vertical lines in between
            \textbf{Database} & \textbf{1st Filter} & \textbf{2nd Filter} & \textbf{3rd Filter} \\
            \hline
            Scopus            & 56                  & 48                  & 35                  \\
            \hline
            Proquest          & 30                  & 24                  & 19                  \\
            \hline
            ACM Digital       & 3                   & 2                   & 2                   \\
            \hline
            Google Scholar    & 13                  & 9                   & 8                   \\
            \hline
            IEEE Explore      & 9                   & 7                   & 5                   \\
        \end{tabular}
    \end{center}
\end{table}




\subsubsection{Scopus}
\noindent
TITLE-ABS-KEY("mining" OR "classifi*") and TITLE-ABS-KEY("association rules") and TITLE-ABS-KEY("questionnaire" or "poll" or "census" OR "canvass" OR "survey") and TITLE-ABS-KEY("fuzzy" or "non crisp") AND (PUBYEAR $>$ 2004)

\subsubsection{Proquest}
\noindent
(ti("mining" OR "classifi*") OR abs("mining" OR "classifi*")) AND (ti("association rules") OR abs("association rules")) AND (ti("questionnaire" OR "poll" OR "census" OR "canvass" OR "survey")) OR abs("questionnaire" OR "poll" OR "census" OR "canvass" OR "survey")) AND ("fuzzy" OR "non crisp")

\noindent
with a filter of on or after 1/1/2005

\subsubsection{ACM Digital}
\noindent
\textit{Edit Query}

\noindent
(Title:("mining" OR "classifi*") OR Abstract:("mining" OR "classifi*")) AND (Title:("association rules") OR Abstract:("association rules")) AND (Title:("questionnaire" OR "poll" OR "census" OR "canvass" OR "survey") OR Abstract:("questionnaire" OR "poll" OR "census" OR "canvass" OR "survey")) AND AllField:("fuzzy" OR "non crisp")

\noindent
\textit{Full Query Syntax}

\noindent
"query": { (Title:("mining" OR "classifi*") OR Abstract:("mining" OR "classifi*")) AND (Title:("association rules") OR Abstract:("association rules")) AND (Title:("questionnaire" OR "poll" OR "census" OR "canvass" OR "survey") OR Abstract:("questionnaire" OR "poll" OR "census" OR "canvass" OR "survey")) AND AllField:("fuzzy" OR "non crisp") }
"filter": { Publication Date: (01/01/2005 TO 12/31/2020), ACM Content: DL, NOT VirtualContent: true }

\subsubsection{Google Scholar}
\noindent
The search options in scholar are limited compared to other databases and the closest search to previous databases can be performed by sorting by date and selecting 'Abstract' rather than 'Everything'. This unfortunately only shows papers in the current year. The following two searches were performed to look for "mining" or "classification"

\begin{enumerate}
  \item mining "association rules" questionnaire
  \item classification "association rules" questionnaire
\end{enumerate}

\subsubsection{IEEE Explore}
\noindent
("Document Title":"mining"  OR "Document Title":"classifi*" OR "Abstract":"mining" OR "Document Title":"classifi*") AND
("Document Title":"association rules" OR "Abstract":"association rules") AND
(("Document Title":"questionnaire" OR "Document Title":"poll" OR  "Document Title":"census" OR "Document Title":"canvas" OR "Document Title":"survey") OR
("Abstract":"questionnaire" OR "Abstract":"poll" OR  "Abstract":"census" OR "Abstract":"canvas" OR "Abstract":"survey"))  AND
("Full Text .AND. Metadata":"fuzzy" OR "Full Text .AND. Metadata":"non crisp")

\noindent
with a filter of on or after 1/1/2005


\subsection{Study quality assessment}

Along with excluding and including based on criteria the guidelines suggest a more thorough determination of inclusion based on the quality of research papers. This may often be based on the quality of abstract of a paper but as pointed out in the guidelines the quality of abstract in Information Technology often varies widely compared to other fields.







