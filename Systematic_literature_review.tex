\noindent
In order to confirm that this research is both novel and of relevance the methodology that it follows will be one of a systematic literature review (SLR).
\par
This approach was formalised with a set of guidelines produced by Kitchenham and Charters in 2007 and is very useful in confirming that no relevant work is excluded. In order to conduct a review the guidelines suggest the following stages

\begin{itemize}
  \item Identification of research %(See Section 6.1).
  \item Selection of primary studies %(See Section 6.2).
  \item Study quality assessment %(See Section 6.3).
  \item Data extraction and monitoring %(See Section 6.4).
  \item Data synthesis %(See Section 6.5).
\end{itemize}

\noindent
The following section's take each of these stages and describe how they have been applied for this research. Through using this approach we firstly highlight key research in the field and then show how this research benefits our understanding of the field and leads to our choice of research topic.


\subsection{Identification of research}
Identification of research is essential in confirming amongst other things that the researcher has not been biased in what he/she wants to find. It is this rigour that delineates SLR from other approaches.
\par
The following databases have been identified as representing a broad enough source for all relevant literature searches so as to nullify any question surrounding researcher bias.

\begin{enumerate}
  \item ACM Digital Library (\href{https://dl.acm.org/}{https://dl.acm.org/})
  \item Google Scholar (\href{https://scholar.google.com/}{https://scholar.google.com/})
  \item IEEE Explore (\href{https://ieeexplore.ieee.org/Xplore/home.jsp}{https://ieeexplore.ieee.org/Xplore/home.jsp})
  \item Science \& Technology\\ (\href{https://www.proquest.com/libraries/government/science-technology/}{https://www.proquest.com/libraries/government/science-technology/})
  \item Scopus (\href{https://www.scopus.com/home.uri}{https://www.scopus.com/home.uri})
\end{enumerate}

These databases were selected as they cover a wide corpus of the literature associated with Information Technology and in particular that of machine learning and artificial intelligence.

% \begin{table}[h!]
%   \begin{center}
%     \caption{Search items and corresponding keywords}
%     \label{tab:table1}
%     \begin{tabular}{l|l} % <-- Alignments: 1st column left, 2nd middle and 3rd right, with vertical lines in between
%       \textbf{Search item}         & \textbf{Keyword}                                                   \\
%       \hline
%       Classification               & data mining, classification                                        \\
%       \hline
%       Association rules            & association rules                                                  \\
%       \hline
%       Questionnaire                & questionnaire, poll, census, canvass, survey                       \\
%       \hline
%       Fuzzy                        & fuzzy, non crisp                                                   \\
%       \hline
%       \multirow{2}{*}{Closed data} & \textit{initially it was hoped to search on only closed data}      \\
%                                    & \textit{but researchers do not categorise work on whether}         \\
%                                    & \textit{it is open or closed so results will be manually filtered} \\
%     \end{tabular}
%   \end{center}
% \end{table}


\begin{table}[h!]
  \begin{center}
    \caption{Search items and corresponding keywords}
    \label{tab:table1}
    \begin{tabular}{l|l} % <-- Alignments: 1st column left, 2nd middle and 3rd right, with vertical lines in between
      \textbf{Search item} & \textbf{Keyword}                    \\
      \hline
      Classification       & data mining, classification         \\
      \hline
      % Association rules            & association rules                                                  \\
      % \hline
      Questionnaire        & questionnaire, poll, census, canvas \\
      \hline
      Closed-ended data    & closed                              \\
    \end{tabular}
  \end{center}
\end{table}


This leads our search string to the following form \textit{("census" OR "application" OR "survey" OR "questionnaire" OR "poll" OR "canvass") AND
  ("closed") AND
  ("medical" OR "health") AND
  ("classification") AND
  ("anomaly" OR "anomalous" OR "imbalance" OR "rarity" OR "exception" OR "oddity" OR "inconsistency" OR "abnormality")}

The large number of references produced from such searches will be recorded and managed through Mendeley Desktop.

Also any relevant information about the searches themselves will be recorded along with the search. This may include information such as date of search, years covered or any specific conditions relating to the search.

\subsection{Selection of primary studies}

\noindent
Inclusion and exclusion criteria will initially be set to the following

\noindent
\textbf{Inclusion:}
\begin{enumerate}
  \item Available online
  \item Article is peer reviewed
  \item Full text is available in English
  \item Article on or after 2014
  \item Can be an academic or commercial project
        % \item CHANGE!!!! Does the article has a segment that argues about four, defined search categories (Table II): Big Data Sources, Big Data Ecosystem, Big data ecosystem challenges, and security and privacy?
\end{enumerate}

\noindent
\textbf{Exclusion:}
\begin{enumerate}
  \item Non English papers
  \item Duplicate studies
  \item Magazines, newspapers, websites, podcasts, blogs
\end{enumerate}

\noindent
The excluded resources will however be maintained so that if during the process of inclusion/exclusion too few works result are presented then the criteria maybe adjusted in order to include other relevant work.

\subsection{Study quality assessment}

Along with excluding and including based on criteria the guidelines suggest a more thorough determination of inclusion based on the quality of research papers. This may often be based on the quality of abstract of a paper but as pointed out in the guidelines the quality of abstract in Information Technology often varies widely compared to other fields.











