
%%%Objective Success

%%%\begin{itemize}
%%	\item I will deliver a theoretical model of RF-optical EITSuccess means a model that accounts for a realistic atomic/molecular system, including field polarisationsand enables verifiable absorption plots to be created.
	%%%\item I will deliver a proof of concept experimental verification of RF-optical EITSuccess means a demonstration of the effect in rubidium.
	%%%\item I will deliver a detailed examination of optical polarization effects in RF-optical EITSuccess means experimental data illustrating predicted differences in absorption profiles when the optical field polarization is altered.
%%%\end{itemize}

%%%Aims

%%%\begin{itemize}
%%%	\item To understand the role that a significant wavelength mismatch between the coupling and probe fields plays in EIT
%%%	\item To build a compact EIT system in which the coupling field is a non-optical field.
%%%	\item To understand if absorption in a gaseous EIT system can be usefully controlled by changing the field polarisation
%%%\end{itemize}

\noindent
%The aims of the project are typically stakeholder requirements

%Write down the aims of your stakeholders.Do the aims and objectives form coherent pairs? If there is not a one-to-one mapping, do you need to adapt either aims or objectives? Do your key-words appear in your aims and objectives?

The main aim of the project is closely related to the most critical stakeholder and industry partner. 

%\textit{How can we use machine learning techniques so that the goal of replacing actual medical assessments with a questionnaire becomes a viable proposition for our industry partner?}

\textit{How can we apply machine learning techniques to a questionnaire to replace the role of high cost medical assessments used in selecting a candidate for a specific job role and yet still avoid the liability risk of an incorrect choice?}

%How can we use machine learning techniques to understand human risk under a specific job role with a questionnaire?… before having high cost medical assessments

\noindent
From the above aim we are able to produce the following research questions:

\begin{enumerate}[label=Question \arabic*:, leftmargin=*]
    \item Is it possible, in a timely manner, to reduce the need for a physical medical assessment for a job role by introducing a suitability predictor using only responses given in a medical questionnaire?

    \item Is it possible to improve upon the suitability predictor by allowing actual medical assessment results to be fed back into the live system? 
    
    \item Would removing rare or anomalous candidates from the pool of candidates create a better suitability predictor?
    
    \item How to analyse and compare the results of repeat medical assessments from the same candidate for different job roles over time? 
    
    \item How to verify and validate the above aims?
\end{enumerate}



%\begin{itemize}
%	\item To build a model that is able to classify a candidate into a small "suitability group" based solely on their medical questionnaire answers.
%	\item To build a model that is able to discover a rare or anomalous candidate solely from their medical questionnaire responses.
%	\item Build the capability for any delivered model(s) to be automatically improved from the actual results from any given physical assessment. 
%	\item To be able to explain why any individual candidate has been selected.
%\end{itemize}

