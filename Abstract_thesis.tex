\thispagestyle{empty}
\begin{abstract}
    Questionnaire answers maybe split into two broad categories of open and closed ended (Marshall, 2005~\cite{marshall2005purpose}). Open-ended give the respondee a blank canvas for
    their answers whereas closed-ended offer a single or restricted set of known possible choices. This paper deals exclusively with closed-ended questionnaires and more specifically medical assessment questionnaires.
    \par
    We will firstly demonstrate how unlike open-ended answers that tend to focus on various NLP techniques the research community has no preferred approach for dealing with closed-ended ones.
    The paper will build upon previous work that has been presented to deal with a varied collection of answer types within a single questionnaire.
    Through the use of fuzzy association rules on these  category, list, number, rank and linguistic rank answer types it is proposed that a potential job candidate be classified into a sliding suitability scale.
    This will result in the requirement for a physical medical assessment for the candidate will be diminished.
    \par
    The results will be further enhanced by applying gradient descent to fine tune any discovered features during classification as well as creating a mechanism whereby any membership functions used during the process can dynamically change over time. A final outcome of the work will enable a candidate that was unsuccessful for one role to be considered against a pool of successful candidates for a separate role.

    % This study attempts to review modern deep learning techniques for class imbalance in the hope of remedying such issues existing within a dataset from a company that UTS is currently collaborating with. Although a great deal of research studies are available into the class imbalance issue using traditional methods this paper also considers the area of deep learning.
    % \par
    % Today's researchers in the field of machine learning are often judged, amongst other things, by the accuracy of their classification. However, this accuracy can be greatly affected when one class within a given dataset vastly outnumbers another. The result of such an imbalance creates a situation where most classifications favour the majority class and so the minority class becomes difficult if not impossible to detect. Although this sounds like an unwanted situation if we judge our classification purely on its accuracy we will have achieved a favourable result as most classifications will indeed be correct. Of course the researcher's credibility is not the only concern with this imbalance as many real world classification problems involve heavily imbalanced datasets. Imagine for instance, attempting to predict whether a patient has a malignant melanoma or a benign lesion. In such a scenario not being able to detect the greatly outnumbered malignant classification would obviously be catastrophic for the patient. This paper attempts to firstly categorise current accepted methods to reduce the imbalance problem and then to highlight the fact that most comparison studies that have been written to date have been applied to the field of computer vision.


    % This paper will apply all comparisons to non image datasets.  
    % Finally some of the approaches will be applied to a specific industrial dataset, which has shown to contain a strong imbalance, in the hope that the findings from previous research aid to a more robust classification. 
    %The company owning the dataset is currently collaborating with UTS in the hope of validating the use of machine learning within their business processes.       

    %For this reason although predictive accuracy could be applied in a reasonably balanced dataset it is more usual to see an ROC curve or something similar applied in an imbalanced situation (Ling \& Li, 1998~\cite{ling1998data}; Drummond \& Holte, 2000~\cite{drummond2000explicitly} Provost \& Fawcett, 2001~\cite{provost2001robust}; Bradley 1997~\cite{bradley1997use}) .  

    \vspace{8pt}
    \noindent Dissertation directed by Associate Professor FH and FD \\
    School of Computer Science

\end{abstract}
